\textit{Shepherding behaviours} are a class of flocking behaviors in which one or more agents (called \textit{shepherds}) try to control the motion of another group of agents (called \textit{flock}) by exerting repulsive forces. A real-life example is sheepdogs guiding flocks of sheep. \textit{Herding} denotes a special type of shepherding behaviour in which the shepherds attempt to steer the flock from a starting point to a target. \cite{lien2003shepherding}

In the context of the course \textit{Collective Behaviour}, we decided to investigate the problem of finding optimal herding strategies. This problem has many engineering applications, such as environmental protection or crowd control \cite{baxter2021simulating}.

Our work builds upon the paper titled \textit{Optimal Shepherding and Transport of a Flock} \cite{ranganathan2022optimal} by A. Ranganathan, A. Heyde, A. Gupta, and L. Mahadevan. This paper models herding as an optimization problem for the shepherd using an agent-based approach. We enhanced the existing model by introducing two modifications: a surrounding fence and additional shepherds. We analyzed the effects of these extensions depending on different behaviours of the flock. Additionally, we compared the results of our model to the results of an existing shepherding algorithm from the literature.