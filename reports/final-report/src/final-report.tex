\documentclass[9pt]{pnas-new}
% Use the lineno option to display guide line numbers if required.
% Note that the use of elements such as single-column equations
% may affect the guide line number alignment. 

% \RequirePackage[english,slovene]{babel} % when writing in slovene
\RequirePackage[slovene,english]{babel} % when writing in english

\usepackage{todonotes}

\templatetype{pnasresearcharticle} % Choose template 
% {pnasresearcharticle} = Template for a two-column research article
% {pnasmathematics} = Template for a one-column mathematics article
% {pnasinvited} = Template for a PNAS invited submission

\selectlanguage{english}
% \selectlanguage{slovene}
% \etal{in sod.} % comment out when writing in english
% \renewcommand{\Authands}{ in } % comment out when writing in english
% \renewcommand{\Authand}{ in } % comment out when writing in english

\newcommand{\set}[1]{\ensuremath{\mathbf{#1}}}
\renewcommand{\vec}[1]{\ensuremath{\mathbf{#1}}}
\newcommand{\uvec}[1]{\ensuremath{\hat{\vec{#1}}}}
\newcommand{\const}[1]{{\ensuremath{\kappa_\mathrm{#1}}}}

\newcommand{\num}[1]{#1}

\graphicspath{{./fig/}}

\title{Collective Behaviour\\ Optimal Shepherding}

% Use letters for affiliations, numbers to show equal authorship (if applicable) and to indicate the corresponding author
\author{Franziska Weber}
\author{Franz Muszarsky}
\author{Kimberley Frings}

\affil{Final Report} 

% Please give the surname of the lead author for the running footer
% \leadauthor{Weber, Muszarsky, Frings} 

% Please add here a significance statement to explain the relevance of your work
% \significancestatement{This is the significance statement.}{add | some | keywords}

% Please include corresponding author, author contribution and author declaration information
%\authorcontributions{Please provide details of author contributions here.}
%\authordeclaration{Please declare any conflict of interest here.}
%\equalauthors{\textsuperscript{1}A.O.(Author One) and A.T. (Author Two) contributed equally to this work (remove if not applicable).}
%\correspondingauthor{\textsuperscript{2}To whom correspondence should be addressed. E-mail: author.two\@email.com}

% Keywords are not mandatory, but authors are strongly encouraged to provide them. If provided, please include two to five keywords, separated by the pipe symbol, e.g:
% \keywords{add | some | keywords} 

\begin{abstract}
Herding denotes a special type of so-called shepherding behaviours in which the shepherds try to steer to flock from a starting point to a target. We investigated the problem of finding optimal herding strategies by building upon an existing agent-based shepherding model. We extended this model by adding a surrounding fence to the environment and by considering the case where multiple shepherds are controlling the flock together. We also implemented an alternative algorithm for multiple shepherds and compared the performance of this algorithm to the one of our model. Our investigations revealed that in most cases the surrounding fence does not influence the shepherding process a lot and that the effect of introducing additional shepherds depends strongly on the behaviour of the flock. 

Our model is publicly available at \url{https://github.com/ki-mberley/Collective-Behaviour}.
\end{abstract}

\dates{\textbf{\today}}
\program{BM-RI}
\vol{2023/24}
\no{CB:GA} % group ID
%\fraca{FRIteza/201516.130}

\begin{document}

% Optional adjustment to line up main text (after abstract) of first page with line numbers, when using both lineno and twocolumn options.
% You should only change this length when you've finalised the article contents.
\verticaladjustment{-2pt}

\maketitle
\thispagestyle{firststyle}
\ifthenelse{\boolean{shortarticle}}{\ifthenelse{\boolean{singlecolumn}}{\abscontentformatted}{\abscontent}}{}

% If your first paragraph (i.e. with the \dropcap) contains a list environment (quote, quotation, theorem, definition, enumerate, itemize...), the line after the list may have some extra indentation. If this is the case, add \parshape=0 to the end of the list environment.
\textit{Shepherding behaviors} are a class of flocking behaviors in which one or more agents (called \textit{shepherds}) try to control the motion of another group of agents (called \textit{flock}) by exerting repulsive forces. A real-life example are sheep dogs guiding flocks of sheep. \textit{Herding} denotes a special type of shepherding behavior in which the shepherds attempt to steer the flock from a starting point to a target. \cite{lien2003shepherding}

In the context of the course \textit{Collective Behavior}, we decided to investigate the problem of finding optimal herding strategies. To begin our work, we have chosen the paper titled \textit{Optimal Shepherding and Transport of a Flock} \cite{ranganathan2022optimal} by A. Ranganathan, A. Heyde, A. Gupta, and L. Mahadevan as a starting point. This paper models herding as an optimization problem for the shepherd using an agent-based approach.

This initial report provides an overview of the present state of our project. We start by giving a concise summary of the relevant parts of our selected paper, followed by a description of our steps involved in executing the existing agent-based model (ABM). Subsequently, we present the current state of our GitHub-Repository and describe our strategy and our goals for the further course of this project.


\section*{Methods}
Our initial focus centered on understanding and replicating the ABM presented in our selected paper. Therefore, we start by giving a concise summary of the relevant parts of this paper.

\subsection{Paper Summary}

The research paper \textit{Optimal Shepherding and Transport of a Flock} explores the techniques that a shepherd can employ to effectively guide a group of animals towards a specific destination. The investigation utilizes an ABM to simulate the behavior of both the shepherd and the flock, with the primary goal of maintaining flock cohesion while achieving the desired movement.

The paper identifies three distinct herding strategies, namely \textit{mustering}, \textit{droving}, and \textit{driving}. Mustering involves the shepherd circling the flock to keep it together, while droving entails the shepherd chasing the flock in the intended direction. Driving, on the other hand, involves the shepherd positioning themselves within the flock and guiding it from the inside. The paper delves into an analysis of the efficiency of these three strategies. The efficiency of a strategy is measured with respect to the three goals (A) moving the mass of the herd to the desired location, (B) not loosing any sheep in the process, and (C) keeping the target and the herd in alignment. 

The findings of the investigation suggest that the optimal herding strategy depends on just two parameters, namely the ratio of the herd size to the shepherd repulsion length and the ratio of herd speed to shepherd speed. The paper comes to the conclusion that droving is the most efficient strategy for managing small herds, while driving proves to be the most effective approach for cohesive, tightly-knit herds. Also, as the shepherd speed increases compared to the herd speed, the optimal strategy changes from droving to mustering.\\

Next, we executed the existing implementation of the ABM shared by our selected paper's authors in their GitHub repository\footnote{\url{https://github.com/arphysics/optimal-shepherding/tree/main/ABM_code}}. In the following subsection, we provide a detailed description of our procedure involved in running the model, along with the necessary steps to generate a plot and a video of the simulation results. All the following commands and actions were performed in a Linux operating system environment (Ubuntu 22.04.3).

\subsection{ABM recreation}

\subsubsection{Running the simulation}
The code of the ABM implementation can be found in the directory \texttt{ABM\_code}. The main file used for running the simulation is \texttt{simulate.cc}. The file \texttt{params.txt} contains the parameters for the simulation (e.g., amount of steps, speed of the herd). Since for the moment we are just focusing on recreating the model, we did not adjust any of these parameters yet. 

During the simulation process, the resulting data is recorded in two distinct output files, namely \texttt{data.txt} and \texttt{costdata.txt}. The former contains the positions of every agent and of the shepherd at each timestep and is the main data file for the simulation. The latter stores the values of the objective function at each timestep which is useful for analysis and debugging.

We successfully downloaded the implementation and executed the simulation by following these straightforward steps:

\begin{verbatim}
    git clone git@github.com:arphysics/optimal-shepherding.git
    cd optimal-shepherding/ABM_code
    cp ../paper_outputs/SI_ABM_videos/config.mk ..
    # replace g++-9 with g++-11 in ../config.mk
    make
    ./simulate
\end{verbatim}

\subsubsection{Creating a plot}
To visualize the simulation results with a plot, we employed the file \texttt{trajectory\_plotter.py} from the directory \texttt{ABM\_code}. The resulting plot is saved as \texttt{output\_plot.pdf}. We started by executing the following steps to run the plotting file:

\begin{verbatim}
    python3 -m pip install numpy Pillow pyparsing matplotlib
    python3 trajectory_plotter.py
\end{verbatim}

\noindent
This first attempt was unsuccessful as we encountered the following error:

\begin{Verbatim}[commandchars=\\\{\}]
\textcolor{red}{
    ValueError: setting an array element with a sequence. The requested array}
\textcolor{red}{
    would exceed the maximum number of dimension of 1.}
\end{Verbatim}

Upon careful inspection, we determined that the error stemmed from an attempt to modify two elements of a list simultaneously. It appears that this operation is not allowed when working with a Python list, but is feasible when using a NumPy array. Consequently, we adjusted the code at six specific locations in the way shown below.\\

\begin{tabular}{ll}
\textit{Initial code} & \texttt{xs[i] = [x[i], x[i] + unit\_perp[0] ]} \\
\textit{Updated version} & \texttt{xs[i] = np.array([x[i], x[i] + unit\_perp[0] ]).flatten()}
\end{tabular}\\

This solved the issue and we obtained the plot shown in figure \ref{fig:output_plot}.

\begin{figure}[h]
    \centering\includegraphics[width=0.9\textwidth]{figures/output_plot.png}\caption{Trajectory of the agents and the shepherd over time}
    \label{fig:output_plot}
\end{figure}

\subsubsection{Creating a video}
We also visualized the simulation results through a movie by using the file \texttt{newest\_visualizer.py} from the directory \texttt{ABM\_code}. The resulting video is saved as \texttt{output\_movie\_clone.mp4}. We performed the following steps to create a video of the results:

\begin{verbatim}
    mkdir test_plots
    sudo apt install -y ffmpeg
    python3 newest_visualizer.py
\end{verbatim}

We encountered the same error here as we did when creating the plot and were able to use the same solution to resolve it. The resulting video can be found in our GitHub repository\footnote{\url{https://github.com/ki-mberley/Collective-Behaviour/tree/main}}.


\section*{Results}
\label{sec:results}
\subsection{Results from the original paper}

The original paper \cite{ranganathan2022optimal} identified three different emerging herding strategies, namely \textit{driving}, \textit{droving}, and \textit{mustering}. Mustering involves the shepherd circling the flock to keep it together while droving entails the shepherd chasing the flock in the intended direction. Driving, on the other hand, involves the shepherd positioning themselves within the flock and guiding it from the inside. The paper comes to the conclusion that the optimal herding strategy depends on just two parameters: the ratio of the herd size to the shepherd repulsion length and the ratio of herd speed to shepherd speed. We managed to recreate these three types of shepherding behaviours in our experiments. The exact parameter values that we used to make each of the three strategies emerge can be found in our GitHub repository.

\subsection{Introduction of a surrounding fence}
The introduction of a surrounding fence ensures that both the dogs and sheep are surrounded by a boundary and prevents them from crossing it. In Figures \ref{fig:one-shepherd-no-fence} and \ref{fig:one-shepherd-fence}, the outcomes are depicted for the herding style driving without and with a fence. As intended, the presence of the fence confines the sheep and dog within its boundaries, while the dog still leads the sheep to the target.


\subsection{Introduction of multiple shepherds}

We analyzed the impact of the number of shepherds and of the parameter \texttt{shepherd\_distance\_penalty} on the duration of the herding process. For this analysis, we removed the randomness from our model by setting a fixed random seed. 

In the case of driving, a single dog needed 35240 timesteps to successfully complete the herding process. The fastest result that we were able to achieve in the two-dog scenario was 38102 timesteps for \texttt{shepherd\_distance\_penalty} = 1. This parameter value leads to one dog mainly staying at the center of the herd and the other one mainly staying outside of the herd without influencing it much. With three dogs the number of timesteps reduced to 27260 for \texttt{shepherd\_distance\_penalty} = 1. Again, one dog mostly stayed at the center of the herd while the other two dogs stayed outside of it.

In the case of droving, a single dog needed 1568 timesteps for a successful completion of the herding process. Introducing a second dog with \texttt{shepherd\_distance\_penalty} = 0.01 reduced the duration to 854 timesteps. The two dogs collaborate and drove the herd together towards the target. The droving behaviours for one and two dogs are depicted in Figures \ref{fig:single-shepherd}  and \ref{fig:two-shepherd}. With three dogs the number of timesteps reduced even further to 491 for \texttt{shepherd\_distance\_penalty} = 0.001. Again, the three dogs cooperated and drove the herd together.

In the case of mustering, a single dog needed 9002 timesteps to lead the herd to the target. In the two-dog scenario with  \texttt{shepherd\_distance\_penalty} = 0.01, the duration reduced to 2178 timesteps. The two dogs cooperated and used a mix of droving and mustering. The introduction of a third dog, on the other hand, led to a significant increase of the duration to 41962 timesteps for \texttt{shepherd\_distance\_penalty} = 0.001. The three dogs do not really cooperate but all stay at the center of the herd and drive it towards the target.

\subsection{Comparison to the agent-based shepherding model}

For the agent-based shepherding model, we again studied the driving, the droving, and the mustering scenario with one, two, and three dogs. In all cases, the herding task was successfully completed. Interestingly, the duration of the herding did not seem to depend as much on the shepherding behaviour as it was the case for our objective-function-based model.

In the case of driving, the agent-based model needed 28123 timesteps with one, 24322 with two, and 28186 with three dogs. In the case of droving, it took 27749 timesteps with one, 34581 with two, and 33421 timesteps with three dogs. Lastly, in the case of mustering, 24986 timesteps were required with one, 25929 timesteps with two, and 28861 timesteps with three dogs.

\section*{Discussion}
\label{sec:discussion}
% Improvements to be made for next deadline? Lots of testing writing a nice report about it

% Suggestion from the professor: implement one of the existing shepherding algorithms for multiple shepherds and compare the results to the one of our model (this seems a bit too ambitious though)

In this report, we've explained the model and the two additions we made. The basic versions of the extensions seem to be effective, and we've shared the results we obtained.

For the third and final report, we'll explore ways to improve our progress. This involves extensive testing with different parameters to ensure our extensions always work, making adjustments if necessary. For instance, our current testing involved only two shepherds beginning from the same location. However, we plan to explore the option of having a shepherd start from a different location and analyze the outcomes. We may also consider implementing the fence using forces, as the original authors intended. If our current implementation is satisfactory, we'll use the remaining time to implement an existing shepherding algorithm for multiple shepherds and compare the results to our model. This idea was suggested by the professor, and we still need to figure out how to approach it and which paper to use. Ultimately, we'll consolidate all our efforts from the past weeks into one final paper.

\section*{Conclusion and outlook}
\label{sec:conclusion}
In this project, we have built upon the agent-based shepherding model from the paper \textit{Optimal Shepherding and Transport of a Flock} \cite{ranganathan2022optimal} and extended it by introducing a surrounding fence and additional shepherds. Additionally, we have compared our approach to the existing agent-based shepherding model from the paper \textit{Simulating Single and Multiple
Sheepdogs Guidance of a Sheep Swarm} \cite{baxter2021simulating}.

Possible next steps include a more detailed analysis of our extensions. It would be interesting to study whether the three observed shepherding behaviours driving, droving, and mustering also arise for more than one shepherd and, if so, for which sets of parameters. Furthermore, one could extend the model even further, for example by introducing obstacles along the path from the herd to the target.

\acknow{
\textit{Franziska Weber} took care of the GitHub repository, researched existing models with multiple shepherds, wrote the description of the original model, analyzed the effect of introducing multiple shepherds, implemented the alternative agent-based shepherding algorithm and compared it to the optimization-based approach.
\textit{Kimberley Frings} corrected and initially executed the existing implementation, implemented the model extensions with a fence and multiple shepherds, composed all three reports and analyzed the effect of introducing a surrounding fence. \textit{Franz Muszarsky} created the presentation and the video. All three worked on understanding the implementation of the model and on the organization of the project.} 
\showacknow % Display the acknowledgments section

% \pnasbreak splits and balances the columns before the references.
% If you see unexpected formatting errors, try commenting out this line
% as it can run into problems with floats and footnotes on the final page.
%\pnasbreak

\begin{multicols}{2}
\section*{\bibname}
% Bibliography
\bibliography{bibliography}
\end{multicols}

\newpage

\section*{Appendix}
\label{sec:appendix}
\appendix
\begin{figure}[!h]
    \hspace*{-0.5cm}
    \includegraphics[width=0.85\textwidth]{"Third report/figures/driving_one_dog"}
    \caption{Trajectory plot for one shepherd in the driving scenario without a fence.  The colors of the agents indicate their orientation.}
    \label{fig:one-shepherd-no-fence}
\end{figure}

\begin{figure}[!h]
    \hspace*{-0.5cm}
    \includegraphics[width=0.85\textwidth]{"Third report/figures/driving_one_dog_fence"}
    \caption{Trajectory plot for one shepherd in the driving scenario with a fence.}
    \label{fig:one-shepherd-fence}
\end{figure}

\begin{figure}[!h]
    \hspace*{-0.5cm}
    \includegraphics[width=0.9\textwidth]{"Third report/figures/droving_one_dog"}
    \caption{Trajectory plot for one shepherd in the droving scenario.}
    \label{fig:single-shepherd}
\end{figure}

\begin{figure}[!h]
    \hspace*{-0.5cm}
    \includegraphics[width=0.9\textwidth]{"Third report/figures/droving_two_dogs"}
    \caption{Trajectory plot for two shepherds in the droving scenario.}
    \label{fig:two-shepherd}
\end{figure}

\end{document}








