\textit{Shepherding behaviors} are a class of flocking behaviors in which one or more agents (called \textit{shepherds}) try to control the motion of another group of agents (called \textit{flock}) by exerting repulsive forces. A real-life example are sheep dogs guiding flocks of sheep. \textit{Herding} denotes a special type of shepherding behavior in which the shepherds attempt to steer the flock from a starting point to a target. \cite{lien2003shepherding}

In the context of the course \textit{Collective Behavior}, we decided to investigate the problem of finding optimal herding strategies. To begin our work, we have chosen the paper titled \textit{Optimal Shepherding and Transport of a Flock} \cite{ranganathan2022optimal} by A. Ranganathan, A. Heyde, A. Gupta, and L. Mahadevan as a starting point. This paper models herding as an optimization problem for the shepherd using an agent-based approach.

This initial report provides an overview of the present state of our project. We start by giving a concise summary of the relevant parts of our selected paper, followed by a description of our steps involved in executing the existing agent-based model (ABM). Subsequently, we present the current state of our GitHub-Repository and describe our strategy and our goals for the further course of this project.
