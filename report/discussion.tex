In this report we have described the overall topic and the current state of our project about optimal shepherding strategies. We have understood the ABM presented in the paper \textit{Optimal shepherding and transport of a flock} and plan to use it as starting point for our own investigation. We have already run the existing implementation of the ABM and identified the code files which are relevant for our project.

Our initial plan involved the idea of translating the original implementation from C++ to Python. We felt that Python, being a language we are more comfortable with, would facilitate modifying and adjusting the code. However, upon further examination and considering the feedback from fellow students, we have reconsidered this approach. We now believe that such a translation is not necessary, as we can reuse and extend the existing code with a manageable degree of effort. Additionally, the conversion to Python could potentially have a negative impact on the performance of the simulation.

For the kick-off, we did not yet have concrete ideas for extending our chosen paper. Now, as our comprehension of the ABM has improved, we have identified the scenario of having multiple shepherds as a a suitable extension of the paper's content. This scenario is highlighted as an open question for future study by the authors of our selected paper and was also proposed by fellow students.

Until the second report deadline, we will try to extend the ABM implementation such that it supports multiple shepherds. A possible goal for the final report is to investigate which optimal herding strategies emerge in the case of more than one shepherd.
